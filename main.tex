\documentclass[a4paper,12pt,oneside]{report}
\usepackage{OvidiusFMI}
\usepackage{times}
\usepackage{graphicx}
\usepackage{hyperref}
\usepackage{color,xcolor}
\usepackage{amsmath}
\usepackage{framed}
\usepackage{indentfirst}
\usepackage{enumerate}
\usepackage{listings}
\usepackage{amsmath,amsfonts,amssymb,amsthm,epsfig,epstopdf,url,array}
\usepackage{multicol,multirow}
\definecolor{code}{rgb}{0.97,0.97,0.97}
\lstdefinestyle{customc}{
  belowcaptionskip=1\baselineskip,
  backgroundcolor=\color{code},
  breaklines=true,
%  frame=L,
%  xleftmargin=\parindent,
  language=C,
  showstringspaces=false,
  morekeywords={bool,
  				 glutMainLoop, glutIdleFunc, glMatrixMode, glLoadIdentity, glPushMatrix, glPopMatrix, 
  				 glBegin, glEnd, glTranslatef, glRotatef, glScalef, glColor3f, glColor4f, glutSolidCube, glutWireCube, glutSolidSphere,
  				 glutWireSphere, glutSolidCone,glutSetWindowTitle,glutGet,glClear,glutSwapBuffers,glDepthFunc,
  				 glutWireCone, glutSolidTorus, glutWireTorus, glutSolidDodecahedron, glutWireDodecahedron,
  				 glutSolidOctahedron, glutWireOctahedron, glutSolidTetrahedron, glutWireTetrahedron, 
  				 glVertex3f,glVertex2f,glPointSize,
  				 glutSolidIcosahedron, glutWireIcosahedron, glutSolidTeapot, glutWireTeapot,glutReshapeFunc,
  				 glFlush, gluPerspective, glutPostRedisplay, glutInit, glutKeyboardFunc,glutKeyboardUpFunc,
  				 glutInitWindowSize, glutInitWindowPosition, glutInitDisplayMode, glutCreateWindow, glutDisplayFunc,glutPassiveMotionFunc,
  				 glClear,glTexCoord2f,
  				 glEnable, glDisable, glLightfv, glMaterialfv, glCullFace,glViewport,
  				 glFrontFace,glColor3ub, glShadeModel,
  				 glGenLists, glGetFloatv,glGentextures,glTexImage2D,glTexParameteri, free, glDeleteTextures,
  				 glLineStipple, glLineWidth, glBindTexture,glGenTextures,
  				 glNewList, glEndList, glCallList,
  				 glMap1f,glEvalCoord1f,glMapGrid1d,glEvalMesh1,glMap2f,glEvalCoord2f,glMapGrid2f,glEvalMesh2,
  				 gluBeginTrim, gluEndTrim, gluPwlCurve,glHint,
  				 GLUnurbsObj, gluBeginSurface, gluNurbsSurface, gluEndSurface, gluNewNurbsRenderer, 									gluNurbsProperty,gluQuadricNormals,
  				 glNormal3f,
  				 gluQuadricTexture,GLUquadricObj,gluSphere,
  				 glPolygonMode, glBlendFunc,glFogi,glFogiv,glFogfv,
  				 GLfloat, GLdouble, GLint,GLuint, GLushort,GLubyte, glRasterPos2f,
  				 gluBeginCurve, gluNurbsCurve, gluEndCurve,
  				 glOrtho, gluLookAt, glutBitmapCharacter, 
  				 glInitNames, glPushName, glLoadName, glSelectBuffer, glRenderMode,gluPickMatrix, glGetIntegerv, glutMouseFunc,glutMotionFunc,system,
  				 glPushAttrib, glPopAttrib, glMultMatrixf, sprintf, glClearStencil, glStencilFunc,glStencilOp,glStencilMask,glColorMask,glActiveStencilFaceEXT,fprintf},  
%   numbers=left,                    % where to put the line-numbers; possible values are (none, left, right)
  %numbersep=5pt,                   % how far the line-numbers are from the code
  %numberstyle=\tiny\color{code}, % the style that is used for the line-numbers
  basicstyle=\footnotesize\ttfamily,
  keywordstyle=\bfseries\color{green!40!black},
  commentstyle=\itshape\color{purple!40!black},
  identifierstyle=\color{blue},
  stringstyle=\color{orange},
}

\lstset{escapechar=@,style=customc}


\newtheorem{definition}{Defini\c tie}
\newtheorem{proposition}{Propozi\c tie}
\newtheorem{demonstration}{Demonstra\c tie}
\newtheorem{example}{Exemplu}
\newtheorem{theorem}{Teorem\u a}
\newtheorem{solve}{Rezolvare}
\newtheorem{corollary}{Corolar}

\facultatea{Matematic\u a \c si Informatic\u a}
\specializarea{Informatic\u a}
\teza{licen\c t\u a}
\titlu{Licen\c t\u a}
\coordonatorPrincipal{Cosma Lumini\c ta}
\autor{T\u anase Ramona Elena}
\data{2021}

\begin{document}
\maketitle

\pagenumbering{roman}
\tableofcontents

\pagenumbering{arabic}
%
%
%CAPITOLUL 1
%
%
\chapter{Func\c tii convexe pe intervale}

Studiul funcțiilor convexe ale unei variabile reale oferă o imagine excelentă a frumuseții și fascinației matematicii avansate. Cititorul va găsi aici o mare varietate de rezultate bazate pe argumente simple și intuitive care au aplicații remarcabile. În același timp, ele oferă punctul de plecare al generalizări profunde în stabilirea mai multor variabile, care vor fi discutate în capitolele următoare.

\section{Functii convexe la prima vedere}

De-a lungul acestei cărți, litera I va indica un interval nedegenerat (care este, un interval care conține o infinitate de puncte).

\subsection{Definitie}

O functie \(f: I \rightarrow \mathbb{R}\) se numeste convexa daca,
\begin{displaymath}
f \left ( \left ( 1 - \lambda  \right )x + \lambda y \right )\leq \left ( 1 - \lambda  \right ) f_{\left ( x \right )} + \lambda f_{\left ( y \right )} 	\label{eq:1.1} \tag{1.1}
\end{displaymath}

pentru toate punctele \(x\) si \(y\) din \(I\), si toate \(\lambda \in \left [ 0,1 \right ]\). Aceasta se numeste strict convexa daca inegalitate \ref{eq:1.1} este valabila ori de cate ori x si y sunt puncte distrincte si \(\lambda \in \left ( 0,1 \right )\). Daca \(-f\)  este convexa (respective stric convexa), atunci spunem ca f  este concava (respectiv strict concava). Daca \(f\) este si convexa si concava, atunci spunem ca f este functie afina. 


Functiile afine sunt tocmai functiile de forma \(mx + n\), pentru constante potrivite m si n. Una poate demonstra usor faptul ca urmatoarele trei functii sunt convexe (desi nu sunt strict convexe): partea pozitiva \(x^{+} = max \left \{ x,0 \right \}\), partea negative \(x^{-} = max \left \{ -x,0 \right \}\), si valoarea absoluta \(\left | x \right | = max \left \{ -x,x \right \}\). Impreuna cu functiile afine ele ofera elemente de baza ale intregii clase de functii convexe pe intervale. 

\textbf{Lema 191 si 192} 
Calculele simple arata ca functia patratica \(x^{2}\) este strict convexa pe \(\mathbb{R}\) si ca functia radacina patrata \(\sqrt{x}\) este strict concava pe \(\mathbb{R}_{+}\). In multe cazuri de inetres convexitatea este stabilita prin intermediul celei de a doua derivate.

\textbf{Corolar 148}
Alte criterii de convexitate legate de teoria de baza a functiilor convexe vor fi prezentate in cele ce urmeaza. 
Convexitatea unei functii \(f : I\rightarrow \mathbb{R}\), inseamna geometric ca, punctele de pe graficului lui \(f_{\left [ u,v \right ]}\) sunt sub (sau pe) coarda care uneste capetele \(\left ( u , f_{\left ( u \right )} \right )\)  si \(\left ( v , f_{\left ( v \right )} \right )\), pentru toti \(u, v \in I, u < v\); 
Vezi Fig 1.1 . Astfel inegalitatea \ref{eq:1.1} este echivalenta cu 
\begin{displaymath}
	f\left ( x \right )\leq f\left ( u \right ) +\frac{f\left ( v \right )- f\left ( u \right )}{v - u}\left ( x - u \right ) \label{eq:1.2} \tag{1.2}
\end{displaymath}
pentru toti \(x\in \left [  u, v\right ]\), si \(u, v \in I, u < v\). 

%
%
% TODO figura
%
%

Aceasta remarca arata faptul ca functiile convexe sunt majorate de functiile afine pe orice subinterval compact. 

Fiecare functie convexa f este marginita pe fiecare subinterval compact \(\left [ u , v \right ]\) a intervalului pe care este definite. De fapt , \(f\left ( x \right ) \leq  M = max \left \{ f\left ( u \right ), f\left ( v \right ) \right \}\)  pe \(\left [ u , v \right ]\)  si scriind un punct arbitrar \(x\in  \left [ u , v  \right ]\)  ca  \(x= \frac{\left ( u + v \right )}{2} + t\) pentru unii \(t\) cu \(\left | t \right |\leq \frac{\left ( v - u \right )}{2}\) , deduce cu usurinta ca 

\begin{displaymath}
  f\left ( x \right )=  f\left ( \frac{u+v}{2} + t\right )\geq 2 f\left ( \frac{u + v}{2} \right )- f\left ( \frac{u + v}{2} - t\right )\geq 2f\left ( \frac{u+v}{2} \right ) - M
\end{displaymath}

\subsection{Teorema}
O functie convexa \(f: I \rightarrow \mathbb{R}\) este continua in orice punct interior al lui \(I\). 	


\begin{demonstration}
	Presupunem ca \(a\in  int I\)
 si alegem \(\varepsilon > 0\)
 astfel incat \(\left [ a - \varepsilon , a + \varepsilon  \right ] \subset I\).
 Atunci  
\begin{displaymath}
  f\left ( a \right )\leq \frac{1}{2} f\left ( a - \varepsilon  \right ) + \frac{1}{2}f \left ( a + \varepsilon  \right )
\end{displaymath}
si 
\begin{displaymath}
  f\left ( a \pm t\varepsilon  \right )= f\left ( \left ( 1 - t \right ) a + t\left ( a \pm \varepsilon  \right )\right )\leq \left ( 1 - t \right )f\left ( a \right ) + tf\left ( a\pm \varepsilon  \right )
\end{displaymath}
pentru orice \(t\in \left [ 0 , 1 \right ]\). Prin urmare 
\begin{displaymath}
  t\left ( f\left ( a\pm \varepsilon  \right ) - f\left ( a \right ) \right )\geq f\left ( a\pm t\varepsilon  \right )- f\left ( a \right )\geq -t\left ( f\left ( a\mp \varepsilon  \right ) - f\left ( a \right )\right )
\end{displaymath}

care ne conduce la 

\begin{displaymath}
  \left | f\left ( a\pm t\varepsilon  \right )- f\left ( a \right ) \right |\leq t max \left \{ \left | f\left ( a-\varepsilon  \right )- f\left ( a \right ) \right |, \left | f\left ( a+\varepsilon  \right ) - f\left ( a \right )\right | \right \},
\end{displaymath}
 pentru orice \(t\in \left [ 0 , 1 \right ]\). Continuitatea functiei \(f\) este acum clara. 
	Exemple simple precum, \(f\left ( x \right )= 0\) daca \(x\in \left ( 0 , 1 \right )\), si daca \(f\left ( 0 \right )= f\left ( 1 \right ) = 1\), arate faptul ca salturi in sus pot aparea la punctele finale ale intervalului de definire a unei functii convexe. Din fericire, aceste posibile discontinuitati sunt detasabile. 

\end{demonstration}

\subsection{Propozitie}
Daca \(f: \left [ a, b \right ]\rightarrow \mathbb{R}\) este o functie convexa, atunci limitele \(f\left ( a+ \right ) = \lim_{x\rightarrow a, x> a}f\left ( x \right )\)  si \(f\left ( b- \right ) = \lim_{x\rightarrow b, x< b}f\left ( x \right )\) exista in \(\mathbb{R}\) si
\begin{displaymath}
  \tilde{f}\left ( x \right )= \left\{\begin{matrix}
f\left ( a+ \right ) & \\ 
 f\left ( x \right )& \\ 
 f\left ( b- \right )& 
\end{matrix} \begin{matrix}
daca x= a & \\ 
daca x\in \left ( a,b \right ) & \\ 
 daca x= b& 
\end{matrix}\right.
\end{displaymath}
 este o functie convexa continua. 

Rezultatul este o consecinta a urmatoarelor:	

%TODO

\subsection{Lema}

\begin{demonstration}
TODO	
\end{demonstration}


\subsection{Corolar}

\subsection{Teorema}

\begin{demonstration}
TODO	
\end{demonstration}


\subsection{Remarca}

\subsection{Teorema}

\begin{demonstration}
TODO	
\end{demonstration}


\subsection{Corolar}

\subsection{Propozitie}

\subsection{Lema}

\begin{demonstration}
TODO	
\end{demonstration}


\subsection{Teorema}


\section{Inegalitatea lui Young si consecintele sale}

\subsection{Teorema}


\begin{demonstration}
TODO	
\end{demonstration}

\subsection{Teorema}

\begin{demonstration}
TODO	
\end{demonstration}



%
%
%CAPITOLUL 2
%
%


\chapter{\c Siruri convergente de numere reale}

\section{Teorie}

\begin{definition}

Un \c sir \((x_{n})_{n \in \mathbb{N}} \subset \mathbb{R} \), se nume\c ste convergent dac\u a exist\u a \(x \in \mathbb{R}\) astfel \^ inc\^ at:
\(\forall _{\varepsilon } > 0, \in n_{\varepsilon } \in \mathbb{N} \) astfel \^ inc\^ at este satisfacut\u a inegalitatea: \(\left | x_{n}- x \right | \leq \varepsilon \).

\end{definition}

\begin{proposition}

Unicitatea limitei unui \c sir de numere reale
Fie \((x_{n})_{n \in \mathbb{N}} \subset \mathbb{R}\). 
Dac\u a \(\left\{\begin{matrix}
x_{n} \to  x\\ 
x_{n} \to y
\end{matrix}\right.
\) atunci \(x=y.\)

\end{proposition}

\begin{demonstration}
  S\u a presupunem, prin absurd, c\u a \(x \neq  y\). Cum suntem pe \(\mathbb{R}\) \^ inseamn\u a c\u a avem una din situa\c tiile \(x < y\) sau \(y < x\). Pentru a face o alegere, fie \(x < y\) atunci \(y – x > 0\) \c si din defini\c tie pentru \(\varepsilon = \frac{y- x}{2}  > 0\) rezult\u a c\u a, 
\end{demonstration}


\begin{itemize}
  \item \(\exists  n_{1} \in \mathbb{N}\) astfel \^ inc\^ at \(\left | x_{n} - x  \right | < \frac{y - x }{2} , \forall n \geq n_{1} \)
  \item \(\exists  n_{2} \in \mathbb{N}\) astfel \^ inc\^ at \(\left | x_{n} - y  \right | < \frac{y - x }{2} , \forall n \geq n_{2} \)
\end{itemize}

Fie \(n = max (n _{1}, n_{2}) \geq n_{1}, n_{2}.\) Atunci \(\left | x_{n} - x \right | < \frac{y-x}{2}\) \c si \(\left | x_{n} - y  \right | <  \frac{y-x}{2}\) de unde 

\begin{displaymath}
  y-x = \left | y-x \right | = \left | (y-x_{n})+ (x_{n} -x) \right |\leq \left | y-x_{n} \right | + \left | x_{n} - x \right | < \frac{y-x}{2} + \frac{y-x}{2} = y-x
\end{displaymath}


A\c sadar, \(y-x < y-x\), contradic\c tie!

Un rezultat foarte frecvent folosit este ceea ce se nume\c ste ”teorema cle\c stelui”.

\textbf{Teorema cle\c stelui}

Fie \((x_{n})_{n\in \mathbb{N}}, (y_{n})_{n\in \mathbb{N}},(z_{n})_{n\in \mathbb{N}} \) trei \c siruri de numere reale. 
Dac\u a:

\[\left\{\begin{matrix}
x_{n} \leq  y_{n} \leq z_{n}, \forall  n \in \mathbb{N}\\ 
x_{n} \to x, z_{n} \rightarrow x

\end{matrix}\right. \]
Atunci \(y_{n} \to x\).

\begin{demonstration}
  Vom ar\u ata pentru \^ inceput urm\u atoarea inegalitate. Dac\u a \(a \leq x\leq b\) atunci \(\left | x \right | \leq  max (\left | a \right |, \left | b \right |) \). 
Vom folosi propriet\u a\c tile de la modul. Avem:
\end{demonstration}


\begin{displaymath}
  \left | x \right | = \left\{\begin{matrix} x, daca x \geq 0\\ -x, daca x< 0 \end{matrix}\right. \leq \left\{\begin{matrix} b\leq max(b,-b) = \left | b \right |\leq max(\left | a \right |,\left | b \right |) daca x\geq 0\\  -a\leq max(a,-a) = \left | a \right | 
  \leq max (\left | a \right |,\left | b \right |) daca x< 0 \end{matrix}\right. \leq max (\left | a \right |, \left | b \right |)
\end{displaymath}
 

Din \(x_{n} \leq y_{n}\leq z_{n}\), \(\forall n\in \mathbb{N} \) rezult\u a c\u a \(x_{n}-x \leq y_{n}-x \leq z_{n}-x, \forall n\in \mathbb{N}. \)
De aici folosind inegalitatea demonstrat\u a deducem c\u a:

\[\left | y_{n} - x \right |\leq max (\left | x_{n}-x \right |, \left | z_{n} - x \right |),\forall n\in \mathbb{N} \label{eq:1.1} \tag{1.1}\]


Deoarece \(x_{n} \to x, \forall \varepsilon > 0, \exists {n_{\varepsilon }}'\in \mathbb{N}\) astfel \^ inc\^ at pentru \(\forall n \geq {n_{\varepsilon }}'\) este satisfacut\u a inegalitatea \[\left | x_{n}-x \right |< \varepsilon.\label{eq:1.2} \tag{1.2} \]
	
Similar din \(z_{n} \to x, \forall \varepsilon > 0,\exists {n_{\varepsilon }}'' \in \mathbb{N}\) astfel \^ inc\^ at pentru \(\forall n\geq {n_{\varepsilon }}''\) este satisfacut\u a inegalitatea \[\left | z_{n} -x \right |< \varepsilon. \label{eq:1.3} \tag{1.3}\]

Fie acum \(\varepsilon > 0.\) Not\u am \(n_{\varepsilon } = max ({n_{\varepsilon }}', {n_{\varepsilon }}'')\). Fie acum \(n\geq n_{\varepsilon }\). Deoarece \(n_{\varepsilon \geq }{n_{\varepsilon }}'\) iar \(n\geq n_{\varepsilon }\) rezult\u a c\u a \(n\geq {n_{\varepsilon }}'\) \c si din \ref{eq:1.2} rezult\u a c\u a \[\left | x_{n} - x \right |< \varepsilon \label{eq:1.4} \tag{1.4} \]
	
Deoarece \(n_{\varepsilon }\geq {n}''_{\varepsilon }\) iar \(n\geq n_{\varepsilon}\) \c si din \ref{eq:1.3} rezult\u a c\u a  \[\left | z_{n}-x  \right |< \varepsilon \label{eq:1.5} \tag{1.5}\]
	
Din \ref{eq:1.4} \c si \ref{eq:1.5} rezult\u a c\u a

\[ max(\left | x_{n} -x  \right |, \left | z_{n}-x \right |) = \begin{Bmatrix}
\left | x_{n}-x daca  \right |\\ 
\left | z_{n}-x daca  \right |
\end{Bmatrix} < \varepsilon.  \label{eq:1.6} \tag{1.6} \]

Folosind inegalitatea \ref{eq:1.6} din inegalitatea \ref{eq:1.1} deducem c\u a \(\left | y_{n}-x  \right |< \varepsilon\).
 
A\c sadar am demonstrat : \(\forall \varepsilon > 0, \exists n_{\varepsilon \in \mathbb{N}}\) astfel \^ inc\^ at pentru \(\forall n\geq n_{\varepsilon }\) este satisfacut\u a inegalitatea \(\left | y_{n}-x \right |< \varepsilon.\) 

Conform defini\c tiei aceast\u a inegalitate \^ inseamn\u a c\u a \(y_{n} \to y.\) 

\begin{example}
  Fie \(c \in \mathbb{R}\), Consider\u am \c sirul \(x_{n}=c\). Atunci \(\lim_{n \to \infty }x_{n}=c\) sau \(\lim_{n \to \infty }c=c\), limita unei constante este acea constant\u a. 
\end{example}


\begin{demonstration}
  \(\forall n\in \mathbb{N}\) avem \(x_{n} - c = c - c = 0 , \left | x_{n}-c \right |= 0\). De aici deducem c\u a \(\forall \varepsilon > 0, \exists n_{\varepsilon} = 1 \in \mathbb{N}\) astfel \^ inc\^ at pentru \(\forall n\geq n_{\varepsilon }= 1\) este satisfacut\u a inegalitatea \(\left | x_{n}-c \right |= 0< \varepsilon\). 
	Conform defini\c tiei \(\lim_{n \to \infty }x_{n} = c. \)
\end{demonstration}

	
\begin{proposition}
  Dac\u a un \c sir de numere naturale este convergent atunci el este sta\c tionar. 
Fie \((x_{n})_{n\in \mathbb{N}}\) un \c sir de numere naturale. Dac\u a exist\u a \(x\in \mathbb{R}\) astfel \^ inc\^ at \(\lim_{n \to \infty }x_{n}= x\), atunci exist\u a \(k\in \mathbb{N}\) astfel \^ inc\^ at \(x_{n}= x_{k}, \forall n\geq k\).
\end{proposition}

	
Astfel spus scris desf\u a\c surat \c sirul arat\u a astfel:
\begin{displaymath}
x_{1},x_{2},x_{3},x_{4},.........,x_{k-1},x_{k},x_{k},x_{k}........
\end{displaymath}


\begin{demonstration}
  Deoarece \(\lim_{n \to \infty }x_{n}= x\) pentru \(\varepsilon = \frac{1}{2}> 0, \exists n_{\frac{1}{2}}\in \mathbb{N}\) astfel \^ inc\^ at \(\forall n\geq n_{\frac{1}{2}}\) este satisfacut\u a inegalitatea \(\left | x_{n} -x \right |<  \frac{1}{2}\). 
\end{demonstration}
	
S\u a not\u am \(k=n_{\frac{1}{2}}\in \mathbb{N}\) \c si s\u a re\c tinem c\u a \c stim c\u a \(\forall n\geq k \) este satisfacut\u a inegalitatea 

\begin{displaymath}
  \left | x_{n} -x \right |< \frac{1}{2}. \label{eq:2.1} \tag{2.1} 
\end{displaymath}


Fie \(n\geq k\). Rela\c tia \ref{eq:2.1} fiind adev\u arat\u a pentru orice num\u ar \(\geq k\) ea va fi adev\u arat\u a \^ in particular pentru k adic\u a avem 
\begin{displaymath}
  \left | x_{k}-x \right |< \frac{1}{2}. \label{eq:2.2} \tag{2.2}
\end{displaymath}


Dar la noi \(n\geq k\) deci din \ref{eq:2.1} avem \c si 
\begin{displaymath}
  \left | x_{n}-x \right |< \frac{1}{2}.\label{eq:2.3} \tag{2.3}
\end{displaymath}


Avem 
\begin{displaymath}
  \left | x_{n}-x_{k} \right |= \left | (x_{n}-x)+(x-x_{k}) \right |\leq \left | x_{n}-x \right |+\left | x-x_{k} \right |= 
\end{displaymath}
\begin{displaymath}
  =\left | x_{n}-x \right |+ \left | -(x-x_{k}) \right |= \left | x_{n}-x \right |+ \left | x_{k} -x\right |. \label{eq:2.4} \tag{2.4}
\end{displaymath}



Am folosit inegalitatea tringhiului \c si \(\left | -a \right |= \left | a \right |\). Folosind \ref{eq:2.2} \c si \ref{eq:2.3} din \ref{eq:2.4} deducem c\u a 
\begin{displaymath}
  \left | x_{n}-x_{k} \right |< \frac{1}{2}+ \frac{1}{2}= 1. \label{eq:2.5} \tag{2.5} 
\end{displaymath}


Dar \(x_{n}, x_{k}\) sunt numere naturale, \c si deci diferen\c ta lor este un num\u ar \^ intreg adic\u a \(x_{n}- x_{k}\in \mathbb{Z}\). Cum \(\left |x_{n}- x_{k} \right |\geq 0\) iar din \ref{eq:2.5} \(\left |x_{n}- x_{k} \right |< 1\) rezult\u a c\u a \(\left |x_{n}- x_{k} \right |\in \left [ 0,1 \right ]\) deci \(\left |x_{n}- x_{k} \right |\in\mathbb{Z}\cap \left [ 0,1 \right)= \left \{ ....,-n ,....,-2,-1,0,1,2,3,....,n,... \right \}\cap \left [ 0,1 \right )= \left \{ 0 \right \}\) de unde \(\left | x_{n}-x_{k} \right |=0\) adic\u a \(x_{n}-x_{k}=0,x_{n}=x_{k}.\) 

A\c sasar am demonstrat: \(\forall n\geq k avem x_{n}=x_{k},\) ceea ce \^ incheie demonstra\c tia. 

\section{Exerci\c tii}

\begin{enumerate}
 \item Calcula\c ti
\begin{displaymath}
   \lim_{n\to\infty }\left ( \frac{1}{\sqrt{n^{4}+1}}+ \frac{2}{\sqrt{n^{4}+2} } +\frac{3}{\sqrt{n^{4}+3}}+........+\frac{n}{\sqrt{n^{4}+n}}  \right )
\end{displaymath}


\begin{solve}
  Not\u am \( x_{n}= \frac{1}{\sqrt{n^{4}+1}} + \frac{2}{\sqrt{n^{4}+2}}+\frac{3}{\sqrt{n^{4}+3}}+........+\frac{n}{\sqrt{n^{4}+n}} \).
Adic\u a \( x_{n}= \sum_{k=1}^{n}\frac{k}{n^{4}+k}\).
\end{solve}


\^ in continuare proced\u am astfel. De num\u ar\u ator nu ne atingem. Vom lucra cu numitorul, ideea fiind de a se avea acela\c si numitor peste tot. 

Avem 
\begin{displaymath}
  1\leq k\leq n \Rightarrow 
\end{displaymath}
\begin{displaymath}
  \Rightarrow  n^{4}+1 \leq n^{4}+k \leq n^{4}+n \Rightarrow 
\end{displaymath}
\begin{displaymath}
  \Rightarrow  \sqrt{n^{4}+1}\leq \sqrt{n^{4}+k}\leq \sqrt{n^{4}+1} \Rightarrow 
\end{displaymath}
\begin{displaymath}
  \Rightarrow  \frac{1}{\sqrt{n^{4}+1}}\geq \frac{1}{\sqrt{n^{4}+k}}\geq \frac{1}{\sqrt{n^{4}+n}}.
\end{displaymath}


Acum \^ inmul\c tind cu k ob\c tinem 

\begin{displaymath}
  \frac{k}{\sqrt{n^{4}+1}}\geq \frac{k}{\sqrt{n^{4}+k}}\geq \frac{k}{\sqrt{n^{4}+n}} \label{eq:3.1} \tag{3.1} 
\end{displaymath}


\^ in continuare \^ in rela\c tia \ref{eq:3.1} dam lui \(k\) valorile \(1,2,.....,n\). 

Pentru \(k = 1\) rezult\u a:

\begin{displaymath}
  \frac{1}{\sqrt{n^{4}+1}}\geq \frac{1}{\sqrt{n^{4}+k}}\geq \frac{1}{\sqrt{n^{4}+n}} 
\end{displaymath}


Pentru \(k = 2\) rezult\u a:

\begin{displaymath}
  \frac{2}{\sqrt{n^{4}+1}}\geq \frac{2}{\sqrt{n^{4}+2}}\geq \frac{2}{\sqrt{n^{4}+n}}
\end{displaymath}


Adun\^ and inegalit\u a\c tile de mai sus ob\c tinem 

\begin{displaymath}
  \frac{1}{\sqrt{n^{4}+1}}+ \frac{2}{\sqrt{n^{4}+1}}+......+ \frac{n}{\sqrt{n^{4}+1}} \geq \frac{1}{\sqrt{n^{4}+1}}+ \frac{2}{\sqrt{n^{4}+2}}+......+ \frac{n}{\sqrt{n^{4}+n}}\geq
\end{displaymath}
\begin{displaymath}
  \geq\frac{1}{\sqrt{n^{4}+n}}+ \frac{2}{\sqrt{n^{4}+n}}+......+ \frac{n}{\sqrt{n^{4}+n}}
\end{displaymath}


Sau
\begin{displaymath}
  \frac{1+2+....+n}{\sqrt{n^{4}+1}}\geq x_{n}\geq \frac{1+2+.....+n}{\sqrt{n^{4}+n}}
\end{displaymath}


Dar \c stim c\u a \(1+2+...+n = \frac{n(n+1)}{2}\), deci vom ob\c tine 
\begin{displaymath}
  \frac{n(n+1)}{2\sqrt{n^{4}+1}}\geq x_{n}\geq \frac{n(n+1)}{2\sqrt{n^{4}+n}} \label{eq:3.2} \tag{3.2}
\end{displaymath}

Acum 
\begin{displaymath}
  \lim_{n \to \infty }\frac{n(n+1)}{2\sqrt{n^{4}+1}}=\frac{1}{2}  
\end{displaymath}
\c si 
\begin{displaymath}
    \lim_{n \to \infty }\frac{n(n+1)}{2\sqrt{n^{4}+n}}=\frac{1}{2} \label{eq:3.3} \tag{3.3}
\end{displaymath}

Vom da la ambele factor comun for\c tat. 

Din \ref{eq:3.2} \c si \ref{eq:3.3} \c si teorema cle\c stelui rezult\u a c\u a:
\begin{displaymath}
  \lim_{n \to \infty }x_{n}=\frac{1}{2}
\end{displaymath}
\end{enumerate}


\chapter{\c Siruri m\u arginite}

\section{Teorie}

\begin{definition}
  Fie \((x_{n})_{n\in \mathbb{N}}\) un \c sir de numere reale. \c sirul \((x_{n})_{n\in \mathbb{N}}\) se nume\c ste m\u arginit dac\u a \c si numai dac\u a \(\exists  a, b \in \mathbb{R}, a< b\) astfel \^ inc\^ at \(\forall n\in \mathbb{N}\) este satisfacut\u a inegalitatea \(x_{n}\in \left [ a,b \right ]\), sau echivalent \(\exists M> 0\) astefle \^ inc\^ at \(\forall  n\in \mathbb{N}\) este satisfacut\u a inegalitatea \(\left | x_{n} \right |\leq M\).
\end{definition}

\begin{definition}
  Fie \((x_{n})_{n\in \mathbb{N}}\) un \c sir de numere reale. Spunem c\u a \(\lim_{n \to \infty }x_{n}=\infty\) dac\u a, \(\forall \varepsilon > 0,\exists n_{\varepsilon }\in \mathbb{N}\) astfel \^ inc\^ at pentru \(\forall n\geq n_{\varepsilon }\) este satisfacut\u a inegalitatea \(x_{n}> \varepsilon\). 
Sau \(\forall \varepsilon > 0,\exists n_{\varepsilon }\in \mathbb{N}\) astfel \^ inc\^ at \(x_{n}> \varepsilon ,\forall n\geq n_{\varepsilon }\). 
\end{definition}


\begin{proposition}
  Fie \((x_{n})_{n\in \mathbb{N}}\) un \c sir de numere reale. Dac\u a \(\lim_{n \to \infty }x_{n}=\infty\) atunci \( \lim_{n \to \infty }\frac{1}{x_{n}}=0.\) 
\end{proposition}

\begin{demonstration}
  Fie \(\varepsilon > 0\). Deoarece \(\lim_{n \to \infty }x_{n}=\infty\) din defini\c tie aplicat\u a pentru \(\frac{1}{\varepsilon }> 0\) rezult\u a c\u a \(\exists n_{\varepsilon }\in \mathbb{N}\) astfel \^ inc\^ at pentru \(\forall n\geq n_{\varepsilon }\) este satisfacut\u a inegalitatea \(x_{n}> \frac{1}{\varepsilon }\).
\end{demonstration}
 

Din aceast\u a inegalitate rezult\u a c\u a \(\forall n\geq n_{\varepsilon }\) este satisfacut\u a inegalitatea \(x_{n}> 0\), prin urmare are sens frac\c tia \(\frac{1}{x_{n}}, \forall n\geq n_{\varepsilon }\). Dar inegalitatea de mai sus este echivalent\u a cu \(\exists n_{\varepsilon }\in \mathbb{N}\) astfel \^ inc\^ at \(\forall n\geq n_{\varepsilon }\) este satisfacut\u a inegalitatea\( \frac{1}{x_{n}}< \varepsilon.\) Conform defini\c tiei aceasta \^ inseamn\u a c\u a \(\lim_{n \to \infty }\frac{1}{x_{n}}=0\).


\textbf{Lema Stolz-Cesaro (Cazul \(\frac{1}{\infty }\))}

Fie \(\left ( x_{n} \right )_{n\in \mathbb{N}}\subset \mathbb{R}\) \c si \(\left (\alpha _{n} \right )_{n\in \mathbb{N}}\subset \left ( 0,\infty \right )\) astfel \^ inc\^ at \(\alpha _{n} \uparrow \infty\). 
Dac\u a 
\begin{displaymath}
  \lim_{n \to \infty }\frac{x_{n} - x_{n-1}}{\alpha _{n}-a_{n-1}}\in \mathbb{R}
\end{displaymath}
	atunci 
\begin{displaymath}
  \lim_{n \to \infty }\frac{x_{n}}{ \alpha _{n}}\in \mathbb{R} 
\end{displaymath}
	\c si \^ in plus 
\begin{displaymath}
  \lim_{n \to \infty }\frac{x_{n}}{ \alpha _{n}}= \lim_{n \to \infty }\frac{x_{n} - x_{n-1}}{ \alpha _{n}- \alpha _{n-1}}.
\end{displaymath}

\begin{demonstration}
  Fie \(\alpha = \lim_{n \to \infty }\frac{x_{n}-x_{n-1}}{\alpha _{n}-\alpha _{n-1}}\). 
\end{demonstration}


Atunci \(\forall \varepsilon > 0,\exists n_{\varepsilon }\in \mathbb{N}\) astefl \^ inc\^ at \(\left | \frac{x_{n}-x_{n-1}}{\alpha _{n}- \alpha _{n-1}} - \alpha \right |< \frac{\varepsilon }{2} \forall  n\geq n_{\varepsilon  }\)

Sau, 
\begin{displaymath}
  \alpha _{n} \uparrow, 
\left | x_{n}- x_{n-1 }- \alpha \left ( \alpha _{n}-\alpha _{n-1} \right ) \right | < \frac{\varepsilon }{2}\left ( \alpha _{n}- \alpha _{n-1} \right ), \forall n\geq n_{\varepsilon } \label{eq:4.1} \tag{4.1}
\end{displaymath}


Not\u am cu \(k=n_{\varepsilon }+1\). Pentru \(n\geq k \) lu\^ and \^ in  \ref{eq:4.1} , \(n= k+1, k+2,....,n\) ob\c tinem:
\(\left | x_{k+1} - x_{k} - \alpha \left ( a_{k+1}- a_{k} \right ) \right |< \frac{\varepsilon }{2}\left ( \alpha _{k+1} - \alpha _{k} \right )\). 

\(\left | x_{k+2} - x_{k+1} - \alpha \left ( a_{k+2}- a_{k+1} \right ) \right |< \frac{\varepsilon }{2}\left ( \alpha _{k+2} - \alpha _{k+1} \right )
...
...
...
\left | x_{n} - x_{n-1} - \alpha \left ( a_{n}- a_{n-1} \right ) \right |< \frac{\varepsilon }{2}\left ( \alpha _{n} - \alpha _{n-1} \right )\)

De unde ob\c tinem, prin adunare:

\begin{displaymath}
  \left | x_{n} - x_{k} - \alpha \left ( \alpha _{n}-\alpha _{k} \right ) \right |= 
\end{displaymath}
\begin{displaymath}
  \left | x_{n} - x_{n-1} -\alpha \left ( \alpha _{n}-\alpha _{n-1} \right )+.....+x_{k+2}-x_{k+1}-\alpha \left ( \alpha _{k+2}-\alpha _{k+1} \right )  +x_{k+1}-x_{k}-\alpha \left ( \alpha _{k+1}-\alpha _{k} \right )\right |
\end{displaymath}
\begin{displaymath}
  \leq \left | x_{n}-x_{n-1}-\alpha \left ( \alpha _{n}-\alpha _{n-1} \right ) \right |+.......+\left | x_{k+2}-x_{k+1}-\alpha \left ( \alpha _{k+2}-\alpha _{k+1} \right ) \right |+\left | x_{k+1}-x_{k}- \alpha \left ( \alpha _{k+1}-\alpha _{k} \right ) \right |
\end{displaymath}
\begin{displaymath}
  \leq \frac{\varepsilon }{2}\left ( \alpha _{k+1} -\alpha _{k} \right )+\frac{\varepsilon }{2}\left ( \alpha _{k+2}-\alpha _{k+1} \right )+\frac{\varepsilon }{2}\left ( \alpha _{n}- \alpha _{n-1}\right )= \frac{\varepsilon }{2}\left ( \alpha _{n}-\alpha _{k} \right )
\end{displaymath}

\(\leq \frac{\varepsilon }{2}\alpha _{n}\) deoarece \(\alpha _{k}> 0\). 


\section{Exerci\c tii}

\begin{enumerate}
\item Calcula\c ti 

 Fie \(\alpha > 0\) s\u a se calculeze 

\begin{displaymath}
  \lim_{n \to \infty }\frac{1^{\alpha }+2^{\alpha }+....+n^{\alpha }}{n^{\alpha +1}}
\end{displaymath}


\begin{demonstration}
  Fie \(x_{n}=1^{\alpha }+2^{\alpha }+....+n^{\alpha },a_{n}= n^{\alpha }\). Deoarece \(\alpha > 0 , \alpha \uparrow \infty.\) Din lema Stolz-Cesaro, cazul 
\end{demonstration}

\begin{displaymath}
  \left [ \frac{1}{\infty } \right ], 
\lim_{n \to \infty }\frac{1^{\alpha }+2^{\alpha }+....+n^{\alpha }}{n^{\alpha +1}}=\lim_{n \to \infty }\frac{x_{n}}{\alpha _{n}}= \lim_{n \to \infty } \frac{x_{n+1}-x_{n}}{\alpha _{n+1}-_{n}}=\lim_{n \to \infty } \frac{\left ( n+1 \right )^{\alpha }}{\left ( n+1 \right )^{\alpha+1} -n^{\alpha +1}}
\end{displaymath}
\begin{displaymath}
  \lim_{n \to \infty }\frac{\left ( n+1 \right )^{\alpha +1}-n^{\alpha +1}}{\left ( n+1 \right )^{\alpha }}
\end{displaymath}

D\u am factor comun for\c tat la num\u ar\u ator pe \(n^{\alpha +1}\). Avem 
\begin{displaymath}
  \lim_{n \to \infty }\frac{\left ( n+1 \right )^{\alpha +1}-n^{\alpha +1}}{\left ( n+1 \right )^{\alpha }} = \lim_{n \to \infty }\frac{n^{\alpha +1\left [ \frac{\left ( n+1 \right )^{\alpha +1}}{n^{\alpha +1}} -1\right ]}}{\left ( n+1 \right )^{\alpha }} 
\end{displaymath}
\begin{displaymath}
  = \lim_{n \to \infty }\frac{n^{\alpha }}{\left ( n+1 \right )^{\alpha }}\cdot n\left [ \left ( \frac{n+1}{n} \right )^{\alpha +1}-1 \right ]
\end{displaymath}
\begin{displaymath}
  =\lim_{n \to \infty }\frac{n^{\alpha }}{\left ( n+1 \right )^{\alpha }}\cdot \lim_{n \to \infty }n\left [ \left ( \frac{n+1}{n} \right )^{\alpha +1} -1\right ]
\end{displaymath}
\begin{displaymath}
  =\lim_{n \to \infty }n\left [ \left ( 1+\frac{1}{n} \right )^{\alpha +1}-1 \right ]
\end{displaymath}
\begin{displaymath}
  = \lim_{n \to \infty }\frac{\left ( 1+\frac{1}{n} \right )^{\alpha +1}-1}{\frac{1}{n}}
\end{displaymath}
\begin{displaymath}
  =\lim_{n \to \infty }\frac{\left ( 1+n \right )^{\alpha +1}-1}{n}= \alpha +1
\end{displaymath}

Am folosit limita fundamental\u a 

\begin{displaymath}
  \lim_{x \to \infty }\frac{\left ( 1+x \right )^{\gamma }-1}{x} = \gamma ,\gamma \in \mathbb{R}
\end{displaymath}


\^ intorc\^ andu-ne la problem\u a, ob\c tinem:

\begin{displaymath}
  \lim_{n \to \infty }\frac{1^{\alpha }+2^{\alpha }+....+n^{\alpha }}{n^{\alpha +1}} = \frac{1}{\alpha +1}
\end{displaymath}


\item Calcula\c ti
\begin{displaymath}
  \lim_{n \to \infty }\frac{e^{\sqrt{1}}+ e^{\sqrt{2}}+....+e^{\sqrt{n}}}{\sqrt{n}e^{\sqrt{n}}}
\end{displaymath}


\begin{solve}
  Fie \( x_{n} = e^{\sqrt{1}}+ e^{\sqrt{2}}+......+e^{\sqrt{n}}, a_{n}= \sqrt{n}e^{\sqrt{n}}.\) Avem de calculat \(\lim_{n \to \infty }\frac{x_{n}}{a_{n}}= \left [ \frac{1}{\infty } \right ]\)
\end{solve}



Din lema Stolz-Cesaro 


\begin{displaymath}
  \lim_{n \to \infty }\frac{x_{n}}{a_{n}}= \lim_{n \to \infty }\frac{x_{n+1}-x_{n}}{\alpha _{n+1}-\alpha _{n}}=\lim_{n \to \infty }\frac{e^{\sqrt{n+1}}}{\sqrt{n+1}e^{\sqrt{n+1}}-\sqrt{n}e^{\sqrt{n}}}
\end{displaymath}


Vom calcula acum 
\begin{displaymath}
  \lim_{n \to \infty }\frac{\sqrt{n+1}e^{\sqrt{n+1}}-\sqrt{n}e^{\sqrt{n}}}{e^{\sqrt{n+1}}}
\end{displaymath}


Avem 
\begin{displaymath}
  \lim_{n \to \infty }\frac{\sqrt{n+1}e^{\sqrt{n+1}}-\sqrt{n}e^{\sqrt{n}}}{e^{\sqrt{n+1}}} = \lim_{n \to \infty }\frac{\left ( \sqrt{n+1}-\sqrt{n} \right )e^{\sqrt{n+1}}+ \sqrt{n}\left ( e^{\sqrt{n+1}}-e^{\sqrt{n}} \right )}{e^{\sqrt{n+1}}}
\end{displaymath}

\begin{displaymath}
  = \lim_{n \to \infty }\left ( \sqrt{n+1}-\sqrt{n} \right )+ \lim_{n \to \infty }\frac{\sqrt{n}\left ( e^{\sqrt{n+1}} -e^{\sqrt{n}} \right )}{e^{\sqrt{n+1}}} 
\end{displaymath}


Prima limit\u a, \^ inmul\c tind \c si \^ imp\u ar\c tind cu conjugata ei ne da da 0, adic\u a:
\begin{displaymath}
  \lim_{n \to \infty }\left ( \sqrt{n+1}-\sqrt{n} \right ) = 0
\end{displaymath}


Pentru cea de a doua limit\u a proced\u am astfel:
\begin{displaymath}
  \lim_{n \to \infty }\frac{\sqrt{n}\left ( e^{\sqrt{n+1}} -e^{\sqrt{n}} \right )}{e^{\sqrt{n+1}}}  = \lim_{n \to \infty } \sqrt{n}\left ( 1- \frac{e^{\sqrt{n}}}{e^{\sqrt{n+1}}} \right )
\end{displaymath}

\begin{displaymath}
  =\lim_{n \to \infty }\sqrt{n}\left ( 1-e^{\sqrt{n}-\sqrt{n+1}} \right )
\end{displaymath}

\begin{displaymath}
  = - \lim_{n \to \infty } \frac{e^{\sqrt{n}- \sqrt{n+1}}-1}{\sqrt{n}- \sqrt{n+1}} \cdot \sqrt{n}\left ( \sqrt{n}-\sqrt{n+1} \right )
\end{displaymath}

\begin{displaymath}
  = -\lim_{n \to \infty } \frac{e^{\sqrt{n}- \sqrt{n+1}}-1}{\sqrt{n}- \sqrt{n+1}} \cdot \lim_{n \to \infty }\sqrt{n}\left ( \sqrt{n} -\sqrt{n+1}\right )
\end{displaymath}

\begin{displaymath}
  = -1 \cdot \left ( -\frac{1}{2} \right ) =\frac{1}{2}
\end{displaymath}

Am \^ inmul\c tit \c si am \^ imp\u ar\c tit cu conjugata ei, iar la ultima factor comun for\c tat. 

Din acestea deducem c\u a:
\begin{displaymath}
  \lim_{n \to \infty }\frac{e^{\sqrt{n+1}}}{\sqrt{n+1}e^{\sqrt{n+1}}-\sqrt{n}e^{\sqrt{n}}} = 2
\end{displaymath}


Deci ultima limit\u a din enun\c t \(\lim_{n \to \infty } \frac{x_{n}}{\alpha _{n}} = 2\)
\end{enumerate}

\begin{proposition}
  Fie \(\left ( x_{n} \right )_{n\in \mathbb{N}}\) un \c sir de numere reale strict pozitive. Dac\u a \(\lim_{n \to \infty }\frac{x_{n+1}}{x_{n}}\in \mathbb{R}\) atunci \(\lim_{n \to \infty } \sqrt[n]{x_{n}}\in \mathbb{R}\). \^ in plus \(\lim_{n \to \infty } \sqrt[n]{x_{n}} = \lim_{n \to \infty }\frac{x_{n+1}}{x_{n}}\).
\end{proposition}
 
Pe scurt 

\begin{displaymath}
  \lim_{n \to \infty } \sqrt[n]{x_{n}} = \lim_{n \to \infty }\frac{x_{n+1}}{x_{n}}
\end{displaymath}


\begin{definition}
  Din defini\c tia logaritmilor naturali avem 
\end{definition}

\begin{displaymath}
  \ln x = \alpha  \Leftrightarrow x = e^{\alpha }
\end{displaymath}

De aici deducem c\u a \(x = e^{\alpha } = e^{\ln x}\) adic\u a 
\begin{displaymath}
  x = e^{\ln x}, \forall x> 0
\end{displaymath}

De aici dac\u a \(x=u^{v}\) ob\c tinem \(u^{v} = e^{\ln\left ( u^{v} \right )} = u^{v\ln u}\). 
S\u a re\c tinem aceast\u a egalitate
\begin{displaymath}
  u^{v} = u^{v\ln u}
\end{displaymath}

Ea se folose\c ste tot timpul c\^ and baza \c si puterea sunt variabile. La noi  \(\sqrt[n]{x_{n}}= x_{n}^{\frac{1}{n}} \)de unde folosind egalitatea de mai sus rezult\u a c\u a 
\begin{displaymath}
  \sqrt[n]{x_{n}}= x_{n}^{\frac{1}{n}} = e^{\frac{1}{n}\cdot \ln x_{n}} = e^{\frac{\ln x_{n}}{n}}
\end{displaymath}


Fie \(A = \lim_{n \to \infty }\frac{x_{n}+1}{x_{n}}\). Atunci \(\ln\) fiind func\c tie continu\u a 
\begin{displaymath}
  \ln A = \ln \lim_{n \to \infty }\frac{x_{n}+1}{x_{n}} = \lim_{n \to \infty }\ln \frac{x_{n}+1}{x_{n}}.
\end{displaymath}


Vom ar\u ata c\u a \^ in ipotezele noastre 
\begin{displaymath}
  \lim_{n \to \infty }\frac{\ln x_{n}}{n} = \lim_{n \to \infty }\ln\frac{x_{n+1}}{x_{n}}
\end{displaymath}

 Din lema Stolz-Cesato cazul \(\left [ \frac{1}{\infty } \right ]\), ipotezele sunt satisfacute, rezult\u a c\u a 

\begin{displaymath}
  \lim_{n \to \infty }\frac{\ln x_{n}}{n} 
= \lim_{n \to \infty }\frac{\ln x_{n}}{n}  
= \lim_{n \to \infty}\frac{x_{n+1}-\ln x_{n}}{n+1-n} 
= \lim_{n \to \infty}\left ( \ln x_{n+1} - \ln x_{n} \right ) 
= \lim_{n \to \infty }\ln \frac{x_{n+1}}{x_{n}} 
= \ln A
\end{displaymath}


De aici deducem c\u a 

\begin{displaymath}
  \lim_{n \to \infty }\sqrt[n]{x_{n}} = \lim_{n \to \infty } e^{\frac{\ln x_{n}}{n}} = e^{\lim_{n \to \infty } \frac{\ln x_{n}}{n}} = e^{\ln A} = A = \lim_{n \to \infty}\frac{x_{n+1}}{n}.
\end{displaymath}

\chapter{\c Siruri recurente \c si asimtote oblice}

\section{Teorie}

\begin{theorem}
   Fie \(a\in \mathbb{R}\) \c si \(f: \left ( \alpha ,\infty  \right )n \to \mathbb{R}\) o func\c tie continu\u a cu proprietatea c\u a \(f_{(x)}> x, \forall x > a\). Definim \c sirul de numere reale \(\left ( x_{n} \right )_{n\geq 1}\) prin condi\c tia ini\c tial\u a \(x_{1}> \alpha\) \c si rela\c tia de recuren\c t\u a \(x_{n+1} = f_{\left ( x_{n} \right )}\) pentru orice  \(n\geq 1	\)
\end{theorem}

Atunci 
\begin{displaymath}
  \lim_{x \to \infty }x_{n} = \infty
\end{displaymath}

Dac\u a exist\u a \(b_{0}\in \mathbb{R}\) astfel \^ inc\^ at \(y = x + b_{0}\) este asimtot\u a oblic\u a la graficul func\c tiei f, atunci
\begin{displaymath}
  \lim_{x \to \infty }\frac{x_{n}}{n}=b_{0}
\end{displaymath}

Dac\u a exist\u a \(b_{0}, b_{1}\in \mathbb{R}, b_{0 }\neq 0\) astfel \^ inc\^ at 
\begin{displaymath}
  \lim_{n \to \infty }x\left ( f\left ( x \right )-x-b_{0} \right )= b_{1},
\lim_{n \to \infty } \frac{n}{\ln n}\left ( \frac{x_{n}}{n} -b_{0}\right )=\frac{b_{1}}{b_{0}}
\end{displaymath}


\begin{demonstration}
  Din condi\c tia ini\c tial\u a avem \(x_{1}> \alpha\). Presupunem \(x_{n}> \alpha\).
Din ipoteza \(f_{\left ( x \right )}> x, \forall  x> \alpha\) rezult\u a c\u a \(f_{\left ( x _{n}\right )}> x_{n}\), adic\u a \(x_{n+1}> x_{n}\). Cum Presupunem \(x_{n}> \alpha\) rezult\u a c\u a \(x_{n+1}> \alpha\). Conform induc\c tiei matematice rezult\u a c\u a \(x_{n}> \alpha,  \forall n\geq 1\). 
\end{demonstration}


Fie \(n\geq 1\). Din ipoteza \(f_{\left ( x \right )}> x, \forall x> \alpha\) \c si \(x_{n}> \alpha\) rezult\u a c\u a \(f_{\left ( x_{n} \right )}> x_{n}\) sau \(x_{n+1}> x_{n}\). A\c sadar \c sirul este strict cresc\u ator. Dac\u a prin absurd ar fi majorat, atunci din  teorema lui Weierstrass este convergent \c si fie \(\lim_{n \to \infty }x_{n} = L\in \mathbb{R}\). Cum \c sirul este cresc\u atpr avem \(x_{n}\geq x_{1}, \forall n\geq 1\) de unde, trec\^ and la limit\u a rezult\u a ca \(L \geq 1\). Cum \(x_{1}\geq \alpha\) rezult\u a c\u a \(L\geq \alpha\), iar din ipoteza \(f_{\left ( x \right )}\geq x, \forall x\geq \alpha\) rezult\u a, in particular, \(f_{\left ( L \right )}> L\). Deoarece \(\lim_{n \to \infty }x_{n}= L\), iar f este contnu\u a, rezult\u a c\u a \(\lim_{n \to \infty }f_{\left ( x_{n} \right )} = f_{\left ( L \right )}\) sau \(\lim_{n \to \infty }x_{n+1} = f_{\left ( L \right )}\), adic\u a \(\lim_{n \to \infty }x_{n} = f_{\left ( L \right )}\). Cum \(\lim_{n \to \infty }x_{n} = f_{\left ( L \right )}\), din unicitatea limitei unui \c sir de numere reale rezult\u a c\u a \(f_{\left ( L \right )} = L\), ceea ce este fals. A\c ssadar \c sirul \(\left ( x_{n} \right )_{n\geq 1}\) nu este majorat \c si fiind cresc\u ator, dup\u a cum am demonstrat, \(\lim_{n \to \infty }x_{n}=\infty\). 
Deoarece \(y= x + b_{0}\) este asimtot\u a oblic\u a la graficul func\c tiei f, conform defini\c tiei \(\lim_{x \to \infty }\left ( f_{\left ( x \right )}-x \right )= b_{0}\). Cum, din 1. , \(\lim_{x \to \infty }x_{n} = \infty\), din caracterizarea limitei unei func\c tii \^ intr-un punct cu \c siruri rezult\u a c\u a \(\lim_{x \to \infty }\left ( f_{\left ( x_{n} \right )} -x_{n}\right )= b_{0}\) sau \c tin\^ and cont de rela\c tia de recuren\c t\u a \(\lim_{x \to \infty }\left ( x_{n+1} -x_{n}\right )=b_{0}\). Din lema Stolz-Cesaro, cazul \(\left [ \frac{1}{\infty } \right ],\) rezult\u a c\u a \(\lim_{n \to \infty }\frac{x_{n}}{n}= \lim_{n \to \infty}\left ( x_{n+1} -x_{n}\right ) = b_{0}\).

Pentru orice \(n\geq 1\) not\u am \(y_{n}= x_{n}-b_{n}n\) Avem 
\begin{displaymath}
  y_{n+1}-y_{n}= x_{1}-x_{n}-b_{0} = f_{\left ( x_{n} \right )}-x_{n}-b_{0},\forall n\geq 1.
\end{displaymath}

Cum \(\lim_{x \to \infty }x\left ( f_{\left ( x \right )} -x-b_{0}\right )=b_{1}\) iar din 1. \(\lim_{x \to \infty }x_{n}=\infty\), din caracterizarea limitei unei func\c tii \^ intr-un punct cu \c siruri rezult\u a c\u a \(\lim_{x \to \infty }x_{n}\left ( y_{n+1} -y_{n}\right )=b_{1}\).

Din egalitatea \(f_{\left ( x \right )}-x-b_{0}= x\left ( f_{\left ( x \right )}-x-b_{0} \right )\cdot \frac{1}{x}, \forall x>\alpha,x  \neq 0\) trec\^ and la limit\u a ob\c tinem 
\begin{displaymath}
  f_{\left ( x \right )}-x-b_{0}= \lim_{x \to \infty }x\left ( f_{\left ( x \right )}-x-b_{0} \right )\cdot \lim_{x \to \infty }\frac{1}{x} = b_{1}\cdot 0 = 0.
\end{displaymath}

Adic\u a \(y=x+b_{0}\) este asimtot\u a oblic\u a la graficul func\c tiei f . Din 2. Rezult\u a c\u a \(\lim_{n \to \infty }\frac{x_{n}}{n} = b_{0}\), de unde \c tin\^ and cont c\u a \(b_{0}\neq 0\) rezult\u a c\u a \(\lim_{n \to \infty }\frac{n}{x_{n}} = \frac{1}{b_{0}}\). 


Din egalitatea
 \begin{displaymath}
  \frac{y_{n+1} - y_{n}}{\frac{1}{n}} = x_{n}\left ( y_{n+1} -y_{n}\right )\cdot \frac{n}{x_{n}}, \geq 1
\end{displaymath}

Trec\^ and la lmit\u a ob\c tinem 
\begin{displaymath}
  \lim_{n \to \infty }\frac{y_{n+1} - y_{n}}{\frac{1}{n}} = \frac{b_{1}}{b_{0}}
\end{displaymath}
.

Iar\u a\c si din lema Stolz-Cesaro ob\c tinem 
\begin{displaymath}
  \lim_{n \to \infty }\frac{y_{n}}{1+\frac{1}{2}+...+\frac{1}{n-1}} = \frac{b_{1}}{b_{0}} . 
\end{displaymath}

Cum \(\lim_{n \to \infty }\frac{{1+\frac{1}{2}+...+\frac{1}{n-1}}}{\ln n} = 1\), rezult\u a c\u a \(\lim_{n \to \infty }\frac{y_{n}}{\ln n } = \frac{b_{1}}{b_{0}}\), sau \(\lim_{n \to \infty }\frac{x_{n}-b_{0}n}{\ln n }  = \frac{b_{1}}{b_{0}}\).


O prim\u a aplica\c tie a tepremei o constituie:

\begin{theorem}
  Fie \(\varphi : \left [ 0,\infty  \right ) \to \mathbb{R}\) o func\c tie continu\u a cu proprietatea c\u a \(\varphi_{\left ( x \right )}> 0, \forall x> 0\). Definim \c sirul de numere reale \(\left ( x_{n} \right )_{n\geq 1}\) prin condi\c tia ini\c tial\u a \(x_{1}> 0\) \c si rela\c tia de recuren\c t\u a \(x_{n+1} = x_{n} + \varphi \left ( \frac{1}{x_{n}} \right )\) pentru \(\forall  n\geq 1\). 
\end{theorem}

Atunci :
\(\lim_{x \to \infty } x_{n} = \infty\) \c si \(\lim_{x \to \infty }\frac{x_{n}}{n} = \varphi\left ( 0 \right )\) iar dac\u a \^ in plus , \(\varphi\left ( 0 \right )> 0\) \c si  \(\varphi\) este derivabil\u a \^ in 0, 
\begin{displaymath}
  \lim_{n \to \infty }\frac{n}{\ln n }\left ( \frac{x_{n}}{n} -\varphi \left ( 0 \right )\right ) = \frac{{\varphi }'\left ( 0 \right )}{\varphi \left ( 0 \right )}.
\end{displaymath}
 

\begin{demonstration}
  Fie \(f : \left ( 0,\infty  \right ) \to \mathbb{R}, f_{\left ( x \right )} = x+ \varphi \left ( \frac{1}{x} \right )\). Evident f este continu\u a \c si deoarece \(\varphi \left ( x \right )> 0\) rezult\u a c\u a \(f_{\left ( x \right )}> x, \forall x> 0 \). 
\end{demonstration}

Deoarece \(\varphi\) este continu\u a \^ in 0, \(\lim_{x \to \infty }\left ( f_{\left ( x \right ) }-x\right ) = \lim_{x \to \infty }\varphi \left ( \frac{1}{x} \right )  = \lim_{t \to 0, t> 0 }\varphi \left ( t \right ) = \varphi \left ( 0 \right )\). 
A\c sadar \(y= x+\varphi \left ( 0 \right )\) este asimtot\u a oblic\u a la graficul func\c tiei f. Din prima teorem\u a 1 \c si 2 rezult\u a c\u a \(\lim_{n \to \infty }x_{n} = \infty\) \c si \(\lim_{n \to \infty }\frac{x_{n}}{n} = \varphi \left ( 0 \right )\). Deoarece \(\varphi\) este derivabil\u a \^ in 0, \(\lim_{x \to \infty } x\left ( f_{\left ( x \right )}-x- \varphi \left ( 0 \right ) \right ) = \lim_{x \to \infty } x\left ( \varphi \left ( \frac{1}{x} \right ) -\varphi \left ( 0 \right )\right ) = \lim_{t \to 0, t> 0 } \frac{\varphi \left ( t \right )-\varphi \left ( 0 \right )}{t} = {\varphi }'\left ( 0 \right )\). Cum \(\varphi \left ( 0 \right )> 0\), din prima teorem\u a , 3. , rezult\u a c\u a 
\begin{displaymath}
  \lim_{n \to \infty }\frac{n}{\ln n } \left ( \frac{x_{n}}{n }- \varphi \left ( 0 \right ) \right )= \frac{{\varphi}'\left ( 0 \right )}{\varphi \left ( 0 \right )}
\end{displaymath}


A doua aplica\c tie a teoremei o constituie 

\begin{theorem}
  Fie \(\varphi : \left [ 0,\infty  \right ) \to \mathbb{R}\) o func\c tie continu\u a, derivabil\u a \^ in 0 cu proprietatea c\u a \(\varphi \left ( x \right )> 1, \forall x> 0, \varphi \left ( 0 \right ) = 1\). Definim \c sirul de numere reale \(\left ( x_{n} \right )_{n\geq 1}\), prin condi\c tia ini\c tial\u a \(x_{1}> 0\) \c si rela\c tia de recuren\c t\u a \(x_{n+1}= x_{n }\varphi\left ( \frac{1}{x_{n}} \right )\) pentru orice \(n\geq 1\). 
\end{theorem}
Atunci:
\(\lim_{x \to \infty }x_{n} = \infty\) \c si \(\lim_{n \to \infty }\frac{x_{n}}{n} = {\varphi }'\left ( 0 \right )\) iar dac\u a \^ in plus  \({\varphi }'\left ( 0 \right )> 0\) \c si \(\varphi\) este de dou\u a ori derivabil\u a \^ in 0, \(\lim_{n \to \infty }\frac{n}{\ln n}\left ( \frac{x_{n}}{n}-{\varphi }' \left ( 0 \right )\right )= \frac{{\varphi }''\left ( 0 \right )}{2{\varphi }'\left ( 0 \right )}\). 

\begin{demonstration}
  Fie \(f : \left ( 0,\infty  \right ) \to \mathbb{R}, f_{\left ( x \right )}= x\varphi \left ( \frac{1}{x} \right )\). Evident f este continu\u a \c si deoarece \(\varphi \left ( x \right )> 1,\forall x> 0\) rezult\u a c\u a \(f_{\left ( x \right )}> x, \forall x> 0\). Deoarece \(\varphi\) este continu\u a \^ in 0 \c si \(\varphi \left ( 0 \right )= 1\) rezult\u a c\u a \(\lim_{x \to \infty }\frac{f_{\left ( x \right )}}{x} = \lim_{x \to \infty } \varphi \left ( \frac{1}{x} \right ) = \lim_{t \to 0, t> 0 }\varphi \left ( t \right ) = \varphi \left ( 0 \right ) = 1\). Deoarece \(\varphi\) este derivabil\u a \^ in 0, 
\end{demonstration}

\begin{displaymath}
  \lim_{x \to \infty }\left ( f_{\left ( x \right )} -x\right ) = \lim_{x \to \infty } x\left ( \varphi \left ( \frac{1}{x} \right ) -1\right )\lim_{t \to 0, t> 0}\frac{\varphi \left ( t \right ) - \varphi \left ( 0 \right )}{t} = 
\end{displaymath}
\begin{displaymath}
  {\varphi }'\left ( 0 \right )
\end{displaymath}
A\c sadar \(y = x +{\varphi }'\left ( 0 \right )\).
 
este asimtot\u a oblic\u a la graficul func\c tiei f. Din teorema 1, 1 si 2, rezult\u a c\u a \(\lim_{n \to \infty }x_{n} = \infty\) \c si \(\lim_{n \to \infty }\frac{x_{n}}{n} = {\varphi }'\left ( 0 \right )\). Deoarece \(\varphi\) este de dou\u a ori derivabil\u a \^ in 0,
\begin{displaymath}
  \lim_{x \to \infty }x\left ( f_{\left ( x \right ) } -x-{\varphi }'\left ( 0 \right )\right ) = \lim_{x \to \infty }x\left ( x\varphi \left ( \frac{1}{x} \right ) - x - {\varphi }'\left ( 0 \right ) \right ) = \end{displaymath}
\begin{displaymath}
  \lim_{t \to 0, t> 0}\frac{\varphi \left ( t \right ) - \varphi \left ( 0 \right )- {\varphi }'\left ( 0 \right )t}{t^{2}} = \frac{{\varphi }''\left ( 0 \right )}{2}
\end{displaymath}

Din teorema 1, 3, rezult\u a c\u a 
\begin{displaymath}
  \lim_{n \to \infty }\frac{n}{\ln n }\left ( \frac{x_{n}}{n}  - {\varphi }'\left ( 0 \right )\right ) = \frac{{\varphi }''\left ( 0 \right )}{2{\varphi }'\left ( 0 \right )}.
\end{displaymath}
 

\begin{corollary}
  Fie \(\alpha \in \mathbb{R} \setminus \left \{ 0 \right \}\). Definim \c sirul de numere reale \(\left ( x_{n} \right )_{n\geq 1}\) prin condi\c sia ini\c tial\u a \(x_{1}> 0\) \c si rela\c tia de recuren\c t\u a \(x_{n+1}= x_{n}+ e^{\frac{\alpha }{x_{n}}}\) pentru orice \(n\geq 1\).
\end{corollary}
 
Atunci \(\lim_{x \to \infty }x_{n} = \infty , \lim_{n \to \infty }\frac{x_{n}}{n} = 1, \lim_{n \to \infty }\frac{n}{\ln n }\left ( \frac{x_{n}}{n} -1\right ) = \alpha\). 

\begin{demonstration}
  Fie \(\varphi :\left [ 0,\infty  \right ) \to \left ( 0,\infty \right ), \varphi \left ( x \right ) = e^{\alpha x}\). S\u a observ\u am c\u a \({\varphi }'\left ( x \right ) = \alpha e^{\alpha x}\). Aplic\u am de dou\u a ori func\c tia \(\varphi\). 
\end{demonstration}


\begin{corollary}
  Fie \(\alpha > 1,\beta > 0\). Definim \c sirul de numere reale \(\left ( x_{n} \right )_{n\geq 1}\) prin condi\c tia \(x_{1} > 0\) \c si rela\c tia de recuren\c t\u a 
\end{corollary}

\begin{displaymath}
  x_{n+1} = x_{n} + \ln\left ( \alpha +\frac{\beta }{x_{n}} \right ), pentru \forall n\geq 1.
\end{displaymath}


Atunci 
\begin{displaymath}
  \lim_{x \to \infty }=\infty ,\lim_{n \to \infty }\frac{x_{n}}{n} = \ln \alpha , \lim_{n \to \infty }\frac{n}{\ln n}\left ( \frac{x_{n}}{n} - \ln \alpha \right ) = \frac{\beta }{\alpha \ln \alpha }.
\end{displaymath}
 
\begin{demonstration}
  Fie \(\varphi :\left [ 0,\infty  \right ) \to \left ( 0,\infty  \right ), \varphi \left ( x \right ) = \ln \left ( \alpha   + \beta x \right )\). S\u a observ\u am c\u a \({\varphi }'\left ( x \right ) = \frac{\beta }{\alpha +\beta x}\). Aplic\u am teorema 2 pentru func\c tia \(\varphi\).
\end{demonstration}


\begin{corollary}
  Fie \(\alpha > 1,\beta > 0\). Definim \c sirul de numere reale \(\left ( x_{n} \right )_{n\geq 1}\) prin condi\c tia ini\c tial\u a \(x_{1}> 0\) \c si rela\c tia de recuren\c t\u a
\end{corollary}
\begin{displaymath}
  x_{n+1} = x_{n} + \sqrt{\alpha +\frac{\beta }{x_{n}}}, pentru \forall n\geq 1.
\end{displaymath}


Atunci 
\begin{displaymath}
  \lim_{x \to \infty }x_{n} = \infty , \lim_{x \to \infty }\frac{x_{n}}{n} = \sqrt{\alpha }, \lim_{n \to \infty }\frac{n}{\ln n}\left ( \frac{x_{n}}{n} -\sqrt{\alpha }\right ) = \frac{\beta }{2\alpha }. 
\end{displaymath}


\begin{demonstration}
Fie \(\varphi :\left [ 0,\infty  \right ) \to \left ( 0,\infty  \right ), \varphi \left ( x \right ) = \sqrt{\alpha +\beta x}\). S\u a observ\u am c\u a \({\varphi }'\left ( x \right ) = \frac{\beta }{2}\left ( \alpha +\beta x \right )^{-\frac{1}{2}}\). Aplic\u am teorema 2 pentru func\c tia \(\varphi\). 
\end{demonstration}


\section{Exerci\c tii}

Calcula\c ti \begin{displaymath}
  \lim_{n \to \infty }\frac{\sum_{k=1}^{n}k\left ( \sqrt[n]{n+k} -1\right )}{n\ln n } = \frac{1}{2}
\end{displaymath}


\begin{demonstration}
  S\u a not\u am \(x_{n} = \frac{1}{n}\sum_{k=1}^{n} k \ln \left ( n+k \right ), n \geq 1\). 
\end{demonstration}

\begin{displaymath}
  \sum_{k=1}^{n}k\left ( \sqrt[n]{n+k}-1 \right )\sim x_{n}. \label{eq:5.1} \tag{5.1}
\end{displaymath}

Fie \(n\geq 2\). Avem \(\ln \left ( n+1 \right )\leq \ln \left ( n+k \right )\leq \ln \left ( n+n \right ), \forall 1\leq k\leq n,\) de unde 
\begin{displaymath}
  \sum_{k=1}^{n} k \ln \left ( n+1 \right )\leq \sum_{k=1}^{n}k \ln \left ( n+k \right )\leq \sum_{k=1}^{n} k \ln \left ( n+n \right ), \frac{\ln \left ( n+1 \right )}{\ln n }
\end{displaymath}

\begin{displaymath}
  \frac{\sum_{k=1}^{n}k}{n^{2}}\leq \frac{x_{n}}{n\ln n}\leq \frac{\ln \left ( n+n \right )}{\ln n }
\end{displaymath}

\(\frac{\sum_{k=1}^{n}k}{n^{2}}\), sau \^ inc\u a, 
\begin{displaymath}
  \frac{\ln \left ( n+1 \right )}{\ln n} \cdot \frac{n+1}{2n}\leq \frac{x_{n}}{n\ln n }\leq \frac{\ln \left ( n+n \right )}{\ln n }\cdot \frac{n+1}{2n} \label{eq:5.2} \tag{5.2}
\end{displaymath}


Din \ref{eq:5.2} \c si teorema cle\c stelui rezult\u a c\u a \(\lim_{n \to \infty }\frac{x_{n}}{n\ln n } = \frac{1}{2}\). Astfel spus \(x_{n\sim }\frac{n\ln n }{2} (3)\)
Din \ref{eq:5.1} \c si \ref{eq:5.3} rezult\u a c\u a \(\sum_{k=1}^{n}k\left ( \sqrt[n]{n+k}-1 \right )\sim \frac{n\ln n }{2}\), adic\u a egalitatea din enun\c t. 


Calcula\c ti 
\begin{displaymath}
  \lim_{n \to \infty }\frac{\sum_{k=1}^{n}\frac{1}{k\left ( \sqrt[n]{n+k}-1 \right )}}{\frac{\ln^{2}n}{n}} = 1
\end{displaymath}

\begin{demonstration}
  S\u a not\u am \(x_{n}= \frac{1}{n}\sum_{k=1}^{n}\frac{\ln\left ( n+k \right )}{k}, n\geq 1\). 
\c stim c\u a 

\end{demonstration}
\begin{displaymath}
  \sum_{k=1}^{n}\frac{1}{k}\left ( \sqrt[n]{n+k}-1 \right )\sim x_{n}. \label{eq:6.1} \tag{6.1}
\end{displaymath}

Fie \(n\geq 2\). Avem \(\sum_{k=1}^{n}\frac{\ln\left ( n+1 \right )}{k}\leq \sum_{k=1}^{n}\frac{\ln\left ( n+k \right )}{k}\leq \sum_{k=1}^{n}\frac{\ln\left ( n+n \right )}{k}\), de unde 

\(\frac{\ln \left ( n+1 \right )}{n}\sum_{k=1}^{n}\frac{1}{k}\leq x_{n}\leq \frac{\ln \left ( n+n \right )}{n}\sum_{k=1}^{n}\frac{1}{k}\), sau \^ inc\u a, 
\begin{displaymath}
  \frac{\ln \left ( n+1 \right )}{\ln n}\cdot \frac{\sum_{k=1}^{n}\frac{1}{k}}{\ln n}\leq \frac{x_{n}}{\frac{\ln^{2}n}{n}}\leq \frac{\ln \left ( n+n \right )}{\ln n}\cdot \frac{\sum_{k=1}^{n}\frac{1}{k}}{\ln n }. \label{eq:6.2} \tag{6.2}
\end{displaymath}

Cum din lema Stolz- Cesaro, cazul \(\left [ \frac{1}{\infty } \right ], \lim_{n \to \infty }\frac{\sum_{k=1}^{n}\frac{1}{k}}{\ln n} = 1\), din \ref{eq:6.2} \c si teorema cle\c stelui deducem c\u a \(\lim_{n \to \infty }\frac{x_{n}}{\frac{\ln ^{2}n}{n}} = 1\). 
Altfel spus 
\begin{displaymath}
  x_{n}\sim \frac{\ln^{2}n}{n}. \label{eq:6.3} \tag{6.3}
\end{displaymath}


Din \ref{eq:6.1} \c si \ref{eq:6.3} deducem c\u a \(\sum_{k=1}^{n}\frac{1}{k}\left ( \sqrt[n]{n+k}-1 \right )\sim \frac{\ln^{2}n}{n}\), adic\u a egalitatea din enun\c t. 

Calcula\c ti 
\begin{displaymath}
  \lim_{n \to \infty }\frac{\sum_{k=2}^{n}\frac{1}{k \ln k}\left ( \sqrt[n]{n+k}-1 \right )}{\frac{\left ( \ln n \right )\left [ \ln \left ( \ln n  \right ) \right ]}{n}} = 1
\end{displaymath}


\begin{demonstration}
  Not\u am \(x_{n} = \frac{1}{n}\sum_{k=2}^{n}\frac{\ln \left ( n+k \right )}{k \ln k}, \geq 2.\)
\c stim c\u a 

\end{demonstration}
\begin{displaymath}
  \sum_{k=2}^{n}\frac{1}{k\ln k}\left ( \sqrt[n]{n+k}-1 \right )\sim x_{n} . \label{eq:7.1} \tag{7.1}
\end{displaymath}

Fie \(n\geq 2\). Avem \(\sum_{k=2}^{n}\frac{\ln \left ( n+1 \right )}{k\ln k}\leq \sum_{k=2}^{n}\frac{\ln \left ( n+k \right )}{k\ln k}\leq \sum_{k=2}^{n}\frac{\ln \left ( n+1 \right )}{k\ln k}\), de unde,

\(\frac{\ln \left ( n+1 \right )}{n}\left ( \sum_{k=2}^{n}\frac{1}{k\ln k } \right )\leq x_{n}\leq \frac{\ln \left ( n+n \right )}{n}\left ( \sum_{k=2}^{n}\frac{1}{k\ln k } \right )\), sau \^ inc\u a, 
\begin{displaymath}
  frac{\ln \left ( n+1 \right )}{n}\cdot \frac{\sum_{k=2}^{n}\frac{1}{k\ln k }}{\ln \left ( \ln n \right )}\leq \frac{x_{n}}{\frac{\left ( \ln n \right )\left [ \ln\left ( \ln n \right ) \right ]}{n}}\leq \frac{\ln \left ( n+n \right )}{n}\cdot \frac{\sum_{k=2}^{n}\frac{1}{k\ln k }}{\ln \left ( \ln n \right )}. \label{eq:7.2} \tag{7.2}
\end{displaymath}


Din lema Stolz-Cesaro , cazul \(\left [ \frac{1}{\infty } \right ], lim_{n \to \infty }\frac{\sum_{k=1}^{n}\frac{1}{k\ln k}}{\ln\left ( \ln n \right ) } = 1,\) din \ref{eq:7.2} \c si teorema cle\c steluui deducem c\u a \(\lim_{n \to \infty }\frac{x_{n}}{\frac{\left ( \ln n  \right )\left [ \ln\left ( \ln n \right ) \right ]}{n}} = 1\). Altfel spus 
\begin{displaymath}
  x_{n}\sim \frac{\left ( \ln n \right )\left [ \ln\left ( \ln n \right ) \right ]}{n}. \label{eq:7.3} \tag{7.3}
\end{displaymath}

Din \ref{eq:7.1} \c si \ref{eq:7.3} deducem c\u a \(\sum_{k=2}^{n}\frac{1}{k}\left ( \sqrt[n]{n+k}-1 \right )\sim \frac{\left ( \ln n \right )\left [ \ln\left ( \ln n \right ) \right ]}{n}\), adic\u a egalitatea din enun\c t. \cite{dumitru}


\bibliographystyle{unsrt}
\setlength{\baselineskip}{\normalbaselineskip}
\setlength{\parskip}{0pt}
\bibliography{refs}
\end{document}